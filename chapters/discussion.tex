\newpage
\section{Discussion}
\subsection{Stopper movement minimization}
 
As briefly mentioned a specific requirement in this case are specific and for the URS it is specified for the stopper height.

URS/1 fdfdfs

Perhaps a better method of reducing plunger movement is by controlling the height of the plunger as well as the air space between the plunger and the sterile liquid. Studies conducted by Kinney et al. demonstrate that more plunger movement is expected as the air space between the plunger and the sterile liquid increases and as the height of the plunger decreases.

The resistance necessary for the snap-fit to occur is on the one hand provided by the air pressure compressed between liquid and rubber stopper, and on the other through friction of the stopper with the surrounding glass container. The combination of these two factors lead to

\subsubsection{Friction}
Silicone oil distribution should be as homogeneous as possible. Dry zones increase the friction. 	viscosity of Silicone oil and drug product  affect accorded to Formula 3 the friction. Interactions with the silicon layer through expedients are neglected in this work. But in the most practical cases, interactions between silicon oil layer on the barrel and product modify the friction[8]. Storage time affect the silicon amount between stopper and syringe barrel and the migration of silicone oil into the solution. The silicone oil flows out and friction increases over the time. Effect to $d_{oil}$. Silicon layer thickness depends proportional to silicon amount if the wetted surface is constant. Different layer thicknesses change the friction[16]. A thicker layer reduces the static friction. Effect to $d_{oil}$.
\begin{equation}
    d_{oil}=\frac{Silicon amount}{\rho_{oil} d_{barrel} \pi l_{barrel}}
    \end{equation}
Critical factor to BLF are: storage time, storage temperature, syringe size, supplier and silicone amount (show in Figure 11).
Critical for high BLF are a big stopper surface area, high storage temperatures, long storage times and a small silicone amount.
Disadvantage: High BLF can cause problems with the activation of the syringe.
Advantage: The stopper is less sensitive towards mechanical impacts. This means that a stopper movement should not occur.
Possible solutions to lowering the BLF are: Reducing the stopper surface area through a shortened stopper, increase the silicone amount or storage the syringe at 5 C.

\subsubsection{Critical factor contribution}


\subsubsection{Process solution}
\subsubsection{Design solution}

\subsection{Model / experiments}
These can be studied with experiments and model with a certain degree of accuracy. For example, a computer model will have exact dimensions while for the experiment there is always present a certain uncertainty caused by tolerances. 

Difficult to give precise quantitive statements


\subsection{2D - 3D model comparison}
Holistic simulation is too complex
Friction is a tricky business: 
Tribological studies
Silicon oil study
Multiple phenomena at once
The quicker the more interlinked they become

\subsection{Explicit - Static Comparison}

