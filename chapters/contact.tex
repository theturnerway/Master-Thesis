\section{Contact modeling}
The objective of contact analysis is to answer the following questions:
(a) whether two or more bodies are in contact, (b) if they are, where the location
or region of contact is, (c) how much contact force or pressure occurs in the contact
interface, and (d) if there is a relative motion after contact in the interface.

Contact is categorized as boundary nonlinearity, in contrast to both geometric
nonlinearity, which emerges from finite deformation problems, and material
nonlinearity, which is a product of nonlinear constitutive relations. The nonlinearity
of contact can be explained in two aspects. Firstly, if two separate bodies come into
contact, the graph of the contact force vs. displacement looks like a cliff because the
contact force stays at zero when two bodies are separate and increases vertically
after the bodies come into contact. In such a case, a functional relationship is not
available because there is no one-to-one relationship between contact force and
displacement. A similar phenomenon happens in the tangential direction under
friction where two bodies are stuck together until the tangential force reaches a
threshold, after which continuous sliding occurs without further increasing the
tangential force. Such an abrupt change in contact force and slip makes the problem
highly nonlinear. Secondly, in order to be a well-posed problem in mechanics,
either displacement (kinematics) or force (kinetics), but not both, must be given for
every material point. Then, the finite element equation solves for unknown information
with given information. On the displacement boundary, for example, if
displacement is given, reaction force should be calculated. On the other hand, on
the traction boundary, if the applied force is given, the corresponding displacement
is to be calculated. Note that these two boundaries are clearly identified in the
problem definition stage. In the case of contact, however, both displacement and
contact force are unknown, except for very limited cases; that is, the contact
boundary is a part of the solution. The user can only identify a candidate of contact
boundary before solving the problem. Therefore, the finite element analysis procedure
must find (a) whether a material point in the boundary of a body is in contact
with the other body, and if it is in contact, (b) the corresponding contact force must
be calculated. Since the contact force at a material point can affect the deformation
of neighboring points, this process needs to be repeated until finding right states for
all points that are possible in contact. Because of this procedural nature, contact
nonlinearity is often addressed algorithmically (Fig. 5.1).
Frictional, frictionless 
\subsection{Contact Formulations}
Surface description, tolerances, separation, adaptive meshing
\paragraph{Surface description}
\paragraph{Tolerances}
\paragraph{Separation}
\paragraph{Adaptive meshing}
\subsection{Governing equations}


\subsection{Discretization}
Discretization of the contact area into elementary units responsible
for the contact stress transmission from one contacting surface to
another.
Node-to-node discretization
Node-to-segment discretization
Segment-to-segment discretization
Mortar and Nitsche discretizations
Contact domain method for discretization


\subsubsection{Smoothing}
Scheme of two non-matching meshes
Smoothing of the master surface
Contact detection
Contact element construction (edge contact element)
Constructed smoothed contact elements

\subsubsection{Solution}
Boundary value problem with constraints

\subsection{Optimization methods}
Smoothing algorithm 
\subsubsection{Penalty}
\subsubsection{Lagrange multipliers}
\subsubsection{Augmented Lagrangian}

